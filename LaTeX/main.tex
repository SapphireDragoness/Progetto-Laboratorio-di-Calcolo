\documentclass{article}
\usepackage{amsmath}
\usepackage{amsthm}
\usepackage{amsfonts}
\usepackage{amssymb}
\usepackage{graphicx}
\usepackage{subcaption}

\title{Relazione di Laboratorio di Calcolo}
\author{Riccardo Nazzari, Linda Monfermoso}
\date{Primo Semestre 2023/2024}

\begin{document}

\maketitle

\section{Introduzione al progetto}
Il progetto assegnato prevede la scrittura di un programma in linguaggio C per il calcolo di integrali definiti,
utilizzando due metodi diversi e fornendo i risultati con almeno 4 cifre decimali esatte. Abbiamo deciso di determinare
il valore delle funzioni integrate con il \textbf{metodo composito del punto medio} ed il \textbf{metodo composito di Cavalieri/Simpson}.
\section{Richiami alla teoria}
\subsection{Integrazione numerica}
L'integrale è un operatore lineare che, data una funzione f(x) e un intervallo [a, b] nel dominio, restituisce il valore 
dell'area sottesa nel suo grafico. 
\begin{equation*}
   I(f) =  \int_{a}^{b} f(x)\, dx = F(b) - F(a)
\end{equation*}
Questo calcolo risulta facile nel momento in cui siamo a conoscenza della primitiva F, ma non sempre è così: quando la funzione
f(x) viene definita come integrale (come la funzione di Eulero) oppure è nota solo in alcuni punti bisogna ricorrere alla
\textbf{integrazione numerica} dell'integrale.

L'integrazione numerica, detta anche quadratura numerica, consiste in una serie di metodi che stimano il valore di un integrale definito, 
con il vantaggio di poterlo fare senza conoscere la funzione primitiva. I metodi si dividono in due macrocategorie:
\begin{itemize}
   \item \textbf{Formule di Newton-Cotes} di cui fanno parte il punto medio, il metodo Cavalieri/Simpson e la formula del trapezio;
   \item \textbf{Formule di Gauss}.
\end{itemize}

\subsection{Formula del punto medio}
Detta anche \textbf{regola del rettangolo}, è il metodo più semplice per approssimare un integrale definito nella forma:
\begin{equation*}
   \int_{a}^{b} f(x) dx.
\end{equation*}
Questa tecnica non fa altro che approssimare un integrale come un rettangolo formato da:
\begin{itemize}
   \item una base di valore (b-a), dove 'b' e 'a' sono i due estremi di integrazione;
   \item un'altezza di valore f(c), dove 'c' è il punto medio dell'intervallo.
\end{itemize}
Otteniamo quindi un'espressione dell'integrale pari a:
\begin{equation*}
   I_{r} = (b-a)\,f(c)\,=\,(b-a)f \left( \frac{a+b}{2}  \right)
\end{equation*}
\subsubsection{Formula composita del punto medio}
L'errore che si ottiene con il metodo del punto medio è abbastanza alto e si può calcolare come:
\begin{equation*}
   \varepsilon = \int_{a}^{b} f(x) dx - I_r = \frac{(b-a)^3}{24}f''(\varepsilon)
\end{equation*}
dove $\varepsilon$ è un punto compreso nell'intervallo [a, b]. E' possibile tuttavia aumentare l'accuratezza
del metodo nel seguente modo:
\begin{enumerate}
   \item suddividiamo l'intervallo [a, b] in N sottointervalli con la stessa ampiezza:
   \item applichiamo in ciascuno di essi la formula enunciata precedentemente.
\end{enumerate}
\section{Sviluppo del progetto}

\end{document}