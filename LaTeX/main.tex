\documentclass{article}
\usepackage{amsmath}
\usepackage{amsthm}
\usepackage{amsfonts}
\usepackage{amssymb}
\usepackage{graphicx}
\usepackage{subcaption}

\title{Relazione di Laboratorio di Calcolo}
\author{Riccardo Nazzari, Linda Monfermoso}
\date{Primo Semestre 2023/2024}

\begin{document}

\maketitle

\section{Introduzione al progetto}
Il progetto assegnato prevede la scrittura di un programma in linguaggio C per il calcolo di integrali definiti,
utilizzando due metodi diversi e fornendo i risultati con almeno 4 cifre decimali esatte. Abbiamo deciso di determinare
il valore delle funzioni integrate con il \textbf{metodo composito del punto medio} ed il \textbf{metodo composito di Cavalieri/Simpson}.
\section{Richiami alla teoria}
\subsection{Integrazione numerica}
L'integrale è un operatore lineare che, data una funzione f(x) e un intervallo [a, b] nel dominio, restituisce il valore 
dell'area sottesa nel suo grafico. 
\begin{equation*}
   I(f) =  \int_{a}^{b} f(x)\, dx = F(b) - F(a)
\end{equation*}
Questo calcolo risulta facile nel momento in cui siamo a conoscenza della primitiva F, ma non sempre è così: quando la funzione
f(x) viene definita come integrale (come la funzione di Eulero) oppure è nota solo in alcuni punti bisogna ricorrere alla
\textbf{integrazione numerica} dell'integrale.

L'integrazione numerica, detta anche \textit{quadratura numerica}, consiste nello stimare il valore di un integrale definito, 
con il vantaggio di poterlo fare senza conoscere la funzione primitiva. Del resto sappiamo che l'integrale non è altro che il calcolo 
dell'area della superfcie delimitata dal grafico della funzione, dall'asse delle ascisse e dalle rete di equazione x=a e x=b; di fatto, i 
metodi esistenti approssimano il calcolo di tale area mediante diverse tecniche.

I metodi per l'integrazione numerica si dividono in due macrocategorie:
\begin{itemize}
   \item \textbf{Formule di Newton-Cotes}: si basano sulla valutazione dell'integrando in n+1 punti equidistanti e sono consigliate da utilizzare proprio se i 
   valori sono noti. Fanno parte di questa famiglia il \textbf{metodo del punto medio}, il \textbf{metodo di Cavalieri/Simpson} e la \textbf{formula del trapezio};
   \item \textbf{Formule di Gauss}: sono preferibili da utilizzare quando è possibile modificare i punti dove è valutato l'integrando, basandosi 
   sulla conoscenza di n+1 valori della funzione nell'intervallo considerato.
\end{itemize}
Nel nostro progetto abbiamo impiegato il metodo del punto medio e di Cavalieri/Simpson.
\subsection{Metodo del punto medio}
Conosciuto anche come \textbf{metodo dei rettangoli}, è il modo più semplice per approssimare un integrale definito nella forma:
\begin{equation*}
   \int_{a}^{b} f(x) dx.
\end{equation*}
Possiede un grado di precisione molto basso, ed il calcolo si basa sul rappresentare l'integrale come un rettangolo, formato da:
\begin{itemize}
   \item una base di valore (b-a), dove 'b' e 'a' sono i due estremi di integrazione;
   \item un'altezza di valore f(c), dove 'c' è il punto medio dell'intervallo.
\end{itemize}
Otteniamo quindi un'espressione dell'integrale pari a:
\begin{equation*}
   I_{r} = (b-a)\,f(c)\,=\,(b-a)f \left( \frac{a+b}{2}  \right)
\end{equation*}
\subsubsection{Formula composita del punto medio}
L'errore che si ottiene con il metodo del punto medio appena enunciato è abbastanza alto e si può calcolare come:
\begin{equation*}
   \varepsilon = \int_{a}^{b} f(x) dx - I_r = \frac{(b-a)^3}{24}f''(\varepsilon)
\end{equation*}
In particolare, è lampante come l'utilizzo di \textit{un solo} rettangolo lasci spazio ad un errore di precisione abbastanza grossolano rispetto al risultato sperato.

E' possibile aumentare l'accuratezza di questo metodo dividendo l'intervallo in \textit{n} parti con  la stessa ampiezza \textit{h}, calcolata come:
\begin{equation*}
   h = \frac{b-a}{n}
\end{equation*}
da cui ricaviamo i punti di suddivisione:
\begin{equation*}
   x_0 = a,\,x_1 = a+h,\, x_2 = a + 2h,\,...\,x_n = a_n + nh = b.
\end{equation*}
su cui possiamo calcolare i valori della funzione:
\begin{equation*}
   y_0 = f(a),\,y_1 = f(x_1),\,y_2 = f(x_2),\,...,y_{n-1} = f(x_{n-1}),\,y_n = f(b).
\end{equation*}
Quello che otteniamo sono dei rettangoli che hanno come base l'intervallo di suddivisione, mentre come altezza il segmento rappresentato dal valore di \textit{f} calcolato 
nel primo estremo oppure nel secondo.

A questo punto possiamo calcolare l'area applicando la formula vista in precedenza per ogni intervallo, ovvero:
\begin{equation*}
   \frac{(b-a)}{n} \sum_{i=0}^{n-1} f(x_i).
\end{equation*}


\section{Sviluppo del progetto}

\end{document}