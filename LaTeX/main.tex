\documentclass{article}
\usepackage{amsmath}
\usepackage{amsthm}
\usepackage{amsfonts}
\usepackage{amssymb}
\usepackage{graphicx}
\usepackage{subcaption}

\title{Relazione di Laboratorio di Calcolo}
\author{Riccardo Nazzari, Linda Monfermoso}
\date{Primo Semestre 2023/2024}

\begin{document}

\maketitle

\section{Introduzione al progetto}
Il progetto assegnato prevede la scrittura di un programma in linguaggio C per il calcolo di integrali definiti,
utilizzando due metodi diversi e fornendo i risultati con almeno 4 cifre decimali esatte. Abbiamo deciso di determinare
il valore delle funzioni integrate con il \textbf{metodo del punto medio} ed il \textbf{metodo di Cavalieri/Simpson}.
\section{Richiami alla teoria}
\subsection{Definizione di integrale definito}
L'integrale è un operatore lineare che, data una funzione f(x) e un intervallo [a, b] nel dominio, restituisce il valore 
dell'area sottesa nel suo grafico. 
\begin{equation*}
   I(f) =  \int_{a}^{b} f(x)\, dx = F(b) - F(a)
\end{equation*}
Questo calcolo risulta facile nel momento in cui siamo a conoscenza della primitiva F, ma non sempre è così: quando la funzione
f(x) viene definita come integrale (come la funzione di Eulero) oppure f(x) è nota solo in alcuni punti bisogna ricorrere ad una 
\textbf{valutazione numerica} dell'integrale.
\section{Sviluppo del progetto}

\end{document}